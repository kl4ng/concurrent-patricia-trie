\documentclass[conference]{IEEEtran}

\usepackage{cite}


%\usepackage[cmex10]{amsmath}
% A popular package from the American Mathematical Society that provides
% many useful and powerful commands for dealing with mathematics. If using
% it, be sure to load this package with the cmex10 option to ensure that
% only type 1 fonts will utilized at all point sizes. Without this option,
% it is possible that some math symbols, particularly those within
% footnotes, will be rendered in bitmap form which will result in a
% document that can not be IEEE Xplore compliant!
%
% Also, note that the amsmath package sets \interdisplaylinepenalty to 10000
% thus preventing page breaks from occurring within multiline equations. Use:
%\interdisplaylinepenalty=2500
% after loading amsmath to restore such page breaks as IEEEtran.cls normally
% does. amsmath.sty is already installed on most LaTeX systems. The latest
% version and documentation can be obtained at:
% http://www.ctan.org/tex-archive/macros/latex/required/amslatex/math/


%\usepackage{eqparbox}
% Also of notable interest is Scott Pakin's eqparbox package for creating
% (automatically sized) equal width boxes - aka "natural width parboxes".
% Available at:
% http://www.ctan.org/tex-archive/macros/latex/contrib/eqparbox/


\hyphenation{op-tical net-works semi-conduc-tor}

\begin{document}

\title{Concurrent Patricia Trie}

\author{\IEEEauthorblockN{Cole Garner}
\IEEEauthorblockA{School of Electrical Engineering and Computer Science\\
University of Central Florida\\
Orlando, Florida 32816\\
Email: ColeGarner@knights.ucf.edu}
\and
\IEEEauthorblockN{Kevin Lang}
\IEEEauthorblockA{School of Electrical Engineering and Computer Science\\
University of Central Florida\\
Orlando, Florida 32816\\
Email: klang2012@gmail.com}}

\maketitle


\begin{abstract}
The abstract goes here.
\end{abstract}


\section{Introduction}
A Patricia Trie, also known as a Radix tree, is a unique version of a regular tree. The defining feature of any trie, also knownas a digital tree, is that the position of a node in a tree defines the key for that node. A Patricia trie takes this and optimizes the tree by merging any parent node with only one child in order to save space and create a more efficient tree.\cite{Shafiei2013} Because a Patricia Trie is a relatively simple data structure and has many practical uses, it is a good data structure to have an efficient parallelization technique for. 
\par
Our implementation of a parallelized, lock-free Patricia Tree will aim to maximize performance and space-efficiency while removing some of the overhead of previous implementations by fine-tuning the memory managment. We will combining multiple older approaches and taking the benefits of each and combining them. Our implementation will be focused on using compare-and-swap (CAS) operations which will allow for a blocking-free implementation. \cite{Shafiei2013,Brown2014}
\par
The main changes in our implementation from previous ones is that we will be storing the flags in multiple locations to allow for a smaller portion of the tree to be blocked off for some operations, creating less conflicts overall and increasing performance. \cite{Natarajan2014} The other main improvement we will be making is decreasing the overhead of creating and removing flags by compacting their size as much as possible, and more efficiently handling garbage collection . 
\par
Another large difference in our implementation is that we will be using a newer language called Rust and making use of it's unique advantages to help create our more efficient implementation. One benefit Rust offers is that it has an efficient memory-safety property without using garbage collection. However, the largest benefit for our implementation is their unique approach to concurrency using their ownership model that ensures more than one thread can not attempt to write to the same location at the same time. 


\section{Related Works}
Creating high-performance, non-blocking data structures has advanced in recent years. There is work into making generalized data structures using CAS operations. \cite{Brown2013} This work has further been expanded into making generalized techniques for non-blocking trees. \cite{Brown2014} These techniques revolve around using load-link extended (LLX), store-conditional extended (SCX) and validate-extended (VLX) primitives which are generalized techniques of the standard, non-extended versions of the primitives. \cite{Brown2013, Brown2014} The techniques used are very powerful and efficient and help form a basis for some techniques used in our work. 
\par
Earlier, non-generalized implementation of this technique was seen in a few different data structures. The one related to our work is Shafiei's implementation of non-blocking Patricia Tries. \cite{Shafiei2013} This implementation used a binary tree implementation and handled the paralleziation by creating flag objects for operations that keep track of what has to be changed. These flags are very powerful because they let multiple threads work on one operation so one thread is not forced to wait.  Additionally, since the flag is there there is no chance of a portion of the tree becoming unusable if one thread fails in the middle of an operation. \cite{Shafiei2013, Howley2012} This technique is similar to ours, except we will be eliminating some of the overhead in their implementation due to the large amount of flags they created and the large size of each. Additionally we will be aiming to improve to memory management as compared to it. 
\par
Another similar work is an implementation of a lock-free binary search tree by Natarajan and Mittal. Their work also heavily involves CAS operations but the largest difference is that instead of marking the nodes they mark the edges between the nodes.\cite{Natarajan2014} This has interesting applications in that it allows a smaller portion of the tree to be flagged during insert and delete operations and allows for less conflicts on the whole.
\par
Shun and Blelloch showed another alternative approach to parallelization of trees with a multiway Cartesian tree. Theres is slightly unique in that they first create an array and then convert it into a tree. However, despite being different from out project, the algorithims they show in order to generate the tree from the array using parallization techniques warranted study. We looked into their techniques of differentiating what part a particular node is protected in, but ultimately decided the techniques were too far from our own to be much use. \cite{Shun2014}

\section{Rust Language}
Rust is a new programming language developed by Mozilla. Developed in parallel with their new Servo web rendering engine, it is developed from the ground up to support safety, concurrency, and parallelism, and was thus a very attractive candidate for our implementation of the concurrent patricia trie.\cite{MozillaResearch} Specifically, in terms of safety it guarantees no data races, buffer or stack overflow, and null pointer exceptions for most use cases. Due to its heavily static nature, it can validate the compiled program to be free of such errors and thus allows us to leverage this in creating a concurrent patricia trie that has fully managed memory instead of relying on garbage collection. Similarly, it is a language that focuses on speed, which will also help us achieve our desired result of having a patricia trie implementation that out-performs its counterparts. 
\par
However, Rust's approach to memory management, namely it being handled almost entirely by the language without a separate runtime for a garbage collector, has its limits. While this is true for most use cases, it becomes increasingly difficult for the compiler to guarantee this when the complexities of parallel lock-free algorithms come into play. The very nature of these algorithms relies more heavily on complex linearizability logic than the compiler can deduce, which means that more complex paralizability problems must have their own memory management, as is in our case.
\par
The main functionality from Rust's standard library that we will be utilizing in the implementation of the CPT algorithm are the Arc and AtomicPtr types. Arc<T> enables us to share any sort of information between threads safely, but does not directly have functionality to make that shared data mutable. The AtomicPtr<T> class gives usIn response to this, we must wrap Arc<T> around AtomicPtr<T> to get Arc<AtomicPtr<T>>. 


\section{Algorithm Description}
Go into detail about what changes (if any) we make from the original pseudocode due to the nature of the Rust language (for better or for worse). Go into detail about what we do new (hopefully able to implement it with >2 children, or have less Flag objects).


\section{Conclusion}
The conclusion goes here, once we finish the above sections.


\section{Acknowledgment}
Acknowledge orignial paper author here once he (hopefully) provides source code and any other patricia trie implementations we test against for empirical evaluation.


% appendix section for midterm report
\appendix
\subsection{Challenges}
As mentioned in the Rust Language section, a large part of the challenges thus far was trying to understand this somewhat esoteric language. Both of us come from a heavy background of Java, where the strict OOP styling in addition to garbage collection made reasoning about memory allocation and management on the stack or heap easy to reason about. However, Rust has a unique memory model where there is not only the distinction between stack and heap, but also between what can and cannot be shared, and what can and cannot be modified.
\par
Indeed, until recently it was thought that the Arc<T> datatype in the Rust standard library would be sufficient for our needs, and was a motivator for choosing Rust as the language of implementation as well. However, as we became more familiar with this somewhat sparsely-documented language, it became apparent that a lot of the functionality that drew us towards the language actually became less functional as one moves into more complex parallel algorithms and data structures.
\par
Talk about rust-lang and its difficulties, mainly. Some overlap with Rust Language section expected.


\subsection{Completed Tasks}
We have completed practically all research we need in order to not only implement our current vision for this project, which currently is to not only fully implement the structure described by the pseudo-code in the paper we got most of our inspiration from this project from\cite{Shafiei2013}, but also to extend it to >2 children and possibly make further improvements in respects to the flags that facilitate most of the operations on the data structure.
\par
We have also reached out to the author of the aforementioned paper in hopes of obtaining source code to test against in order to see performance gains from using a non-GC and high-performance language, though we have yet to get a response. However, we have found other suitable source codes for this sort of comparison, also written in Java, though seemingly not as up-to-date in terms of utilizing the recent lock-free developments of patricia tries or any tree in academia.\cite{PATSource}
\par
Research complete. Source code links to other implementations we can compare against. Decent chunk of original algorithm done. 


\subsection{Remaining Tasks}
Optimize code once completed, then attempt to modify algorithm to be extended to >2 children OR use less memory.
Also need to run performance comparisons and get source code from others.


% references section
\bibliographystyle{IEEEtran}
\bibliography{IEEEabrv,references}


\end{document}


%\subsection{Subsection Heading Here}
%Subsection text here.


%\subsubsection{Subsubsection Heading Here}
%Subsubsection text here.
