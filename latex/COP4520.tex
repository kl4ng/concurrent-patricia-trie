\documentclass[conference]{IEEEtran}

\usepackage{cite}


%\usepackage[cmex10]{amsmath}
% A popular package from the American Mathematical Society that provides
% many useful and powerful commands for dealing with mathematics. If using
% it, be sure to load this package with the cmex10 option to ensure that
% only type 1 fonts will utilized at all point sizes. Without this option,
% it is possible that some math symbols, particularly those within
% footnotes, will be rendered in bitmap form which will result in a
% document that can not be IEEE Xplore compliant!
%
% Also, note that the amsmath package sets \interdisplaylinepenalty to 10000
% thus preventing page breaks from occurring within multiline equations. Use:
%\interdisplaylinepenalty=2500
% after loading amsmath to restore such page breaks as IEEEtran.cls normally
% does. amsmath.sty is already installed on most LaTeX systems. The latest
% version and documentation can be obtained at:
% http://www.ctan.org/tex-archive/macros/latex/required/amslatex/math/


%\usepackage{eqparbox}
% Also of notable interest is Scott Pakin's eqparbox package for creating
% (automatically sized) equal width boxes - aka "natural width parboxes".
% Available at:
% http://www.ctan.org/tex-archive/macros/latex/contrib/eqparbox/


\hyphenation{op-tical net-works semi-conduc-tor}

\begin{document}

\title{Concurrent Patricia Trie}

\author{\IEEEauthorblockN{Cole Garner}
\IEEEauthorblockA{School of Electrical Engineering and Computer Science\\
University of Central Florida\\
Orlando, Florida 32816\\
Email: ColeGarner@knights.ucf.edu}
\and
\IEEEauthorblockN{Kevin Lang}
\IEEEauthorblockA{School of Electrical Engineering and Computer Science\\
University of Central Florida\\
Orlando, Florida 32816\\
Email: klang2012@gmail.com}}

\maketitle


\begin{abstract}
The abstract goes here.
\end{abstract}


\section{Introduction}
A Patricia Trie, also known as a Radix tree, is a unique version of a regular tree. The defining feature of any trie, also known
as a digital tree, is that the position of a node in a tree defines the key for that node. A Patricia trie takes this and optimizes the tree by 
merging any parent node with only one child in order to save space and create a more efficient tree.\cite{Shafiei2013} \par



\section{Related Works}
Lock free concurrent trees and data structures in general have been progressing in the last couple of years. \cite{Brown2014,Brown2013}
Since there are numerous data structures, the work on concurrent data structures has been spread out among them. However, Shafiei wrote
a paper about Non-blocking Patricia tries \cite{Shafiei2013} that is related to our work. However, in their case they simply used a binary version of the tree 



\section{Conclusion}
The conclusion goes here.




% references section
\bibliographystyle{IEEEtran}
\bibliography{IEEEabrv,references}


\end{document}


%\subsection{Subsection Heading Here}
%Subsection text here.


%\subsubsection{Subsubsection Heading Here}
%Subsubsection text here.