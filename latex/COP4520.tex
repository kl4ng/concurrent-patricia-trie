\documentclass[conference]{IEEEtran}

\usepackage{cite}


%\usepackage[cmex10]{amsmath}
% A popular package from the American Mathematical Society that provides
% many useful and powerful commands for dealing with mathematics. If using
% it, be sure to load this package with the cmex10 option to ensure that
% only type 1 fonts will utilized at all point sizes. Without this option,
% it is possible that some math symbols, particularly those within
% footnotes, will be rendered in bitmap form which will result in a
% document that can not be IEEE Xplore compliant!
%
% Also, note that the amsmath package sets \interdisplaylinepenalty to 10000
% thus preventing page breaks from occurring within multiline equations. Use:
%\interdisplaylinepenalty=2500
% after loading amsmath to restore such page breaks as IEEEtran.cls normally
% does. amsmath.sty is already installed on most LaTeX systems. The latest
% version and documentation can be obtained at:
% http://www.ctan.org/tex-archive/macros/latex/required/amslatex/math/


%\usepackage{eqparbox}
% Also of notable interest is Scott Pakin's eqparbox package for creating
% (automatically sized) equal width boxes - aka "natural width parboxes".
% Available at:
% http://www.ctan.org/tex-archive/macros/latex/contrib/eqparbox/


\hyphenation{op-tical net-works semi-conduc-tor}

\begin{document}

\title{Concurrent Patricia Trie \\ \normalsize https://github.com/kl4ng/concurrent-patricia-trie}

\author{\IEEEauthorblockN{Cole Garner}
\IEEEauthorblockA{School of Electrical Engineering and Computer Science\\
University of Central Florida\\
Orlando, Florida 32816\\
Email: ColeGarner@knights.ucf.edu}
\and
\IEEEauthorblockN{Kevin Lang}
\IEEEauthorblockA{School of Electrical Engineering and Computer Science\\
University of Central Florida\\
Orlando, Florida 32816\\
Email: klang2012@gmail.com}}

\maketitle


\begin{abstract}
This paper shows an implementation of a non-blocking Patricia Trie that combines techniques found in numerous recent implementations. It heavily uses the Compare and Swap operation and uses flags to prevent blocking. Additionally, it implements the flags in multiple locations in order to both increase efficiency and reduce the amount of conflicts. It also uses more efficient memory management and decreased overhead to improve upon its predecessors.
\end{abstract}


\section{Introduction}
A Patricia Trie, also known as a Radix tree, is a unique version of a regular tree. The defining feature of any trie, also known as a digital tree, is that the position of a node in a tree defines the key for that node. A Patricia trie takes this and optimizes the tree by merging any parent node with only one child in order to save space and create a more efficient tree.\cite{Shafiei2013} Because a Patricia Trie is a relatively simple data structure and has many practical uses, it is a good data structure to have an efficient parallelization technique for.
\par
Our implementation of a parallelized, lock-free Patricia Tree will aim to maximize performance and space-efficiency while removing some of the overhead of previous implementations by fine-tuning the memory management. We will combining multiple older approaches and taking the benefits of each and combining them. Our implementation will be focused on using compare-and-swap (CAS) operations which will allow for a blocking-free implementation. \cite{Shafiei2013,Brown2014}
\par
The main changes in our implementation from previous ones is that we will be storing the flags in multiple locations to allow for a smaller portion of the tree to be blocked off for some operations, creating less conflicts overall and increasing performance. \cite{Natarajan2014} The other main improvement we will be making is decreasing the overhead of creating and removing flags by compacting their size as much as possible, and more efficiently handling garbage collection .
\par
We originally planned to use a newer language called Rust to help our implementation by making use of its memory-safety property. However, we decided against this when we discovered that the language would not work in the way we wanted it to for our implementation.
\par
For the actual Patricia tree algorithm we will be using an algorithm similar to the pseudo-code seen in \cite{Shafiei2013}. However, we will be inserting various improvements on this algorithm, with the main one being that flags will also appear in the edges between two nodes to decrease conflicts. \cite{Natarajan2014}


\section{Related Works}
Creating high-performance, non-blocking data structures has advanced in recent years. There is work into making generalized data structures using CAS operations. \cite{Brown2013} This work has further been expanded into making generalized techniques for non-blocking trees. \cite{Brown2014} These techniques revolve around using load-link extended (LLX), store-conditional extended (SCX) and validate-extended (VLX) primitives which are generalized techniques of the standard, non-extended versions of the primitives. \cite{Brown2013, Brown2014} The techniques used are very powerful and efficient and help form a basis for some techniques used in our work.
\par
Earlier, non-generalized implementation of this technique was seen in a few different data structures. The one related to our work is Shafiei's implementation of non-blocking Patricia Tries. \cite{Shafiei2013} This implementation used a binary tree implementation and handled the parallization by creating flag objects for operations that keep track of what has to be changed. These flags are very powerful because they let multiple threads work on one operation so one thread is not forced to wait. Additionally, since the flag is there there is no chance of a portion of the tree becoming unusable if one thread fails in the middle of an operation. \cite{Shafiei2013, Howley2012} This technique is similar to ours, except we will be eliminating some of the overhead in their implementation due to the large amount of flags they created and the large size of each. Additionally we will be aiming to improve to memory management as compared to it.
\par
A slightly different work is an implementation of a lock-free binary search tree by Natarajan and Mittal. Their work also heavily involves CAS operations but the largest difference is that instead of marking the nodes they mark the edges between the nodes.\cite{Natarajan2014} This has interesting applications in that it allows a smaller portion of the tree to be flagged during insert and delete operations and allows for less conflicts on the whole.
\par
Similar to the previous work is another edge-based algorithm for a concurrent binary search tree by Ramachandran and Mittal. This is not a lock-free solution so it is not completely applicable to ours, however it involves edges and has relatively few nodes locked for each operation. \cite{Ramachandran2015} We will be using some of the techniques used here for the different operations to help further reduce the amount of nodes locked by other implementations and reduce the number of conflicts.
\par
Shun and Blelloch showed another alternative approach to parallization of trees with a multiway Cartesian tree. Theres is slightly unique in that they first create an array and then convert it into a tree. However, despite being different from out project, the algorithms they show in order to generate the tree from the array using parallization techniques warranted study. We looked into their techniques of differentiating what part a particular node is protected in, but ultimately decided the techniques were too far from our own to be much use. \cite{Shun2014}
\par
Due to us implementing this in a way with fully managed memory instead of utilizing a garbage collector, we will be utilizing hazard pointers as initially described by Michael in his seminal 2004 paper. \cite{Michael2004} While the original paper was more oriented towards C++, we will instead be modifying an existing implementation that was found publically available for Rust, extending and modifying it for our own use. \cite{CHAMT}

\section{Rust Language and Libraries}
Rust is a new programming language developed by Mozilla. Developed in parallel with their new Servo web rendering engine, it is developed from the ground up to support safety, concurrency, and parallelism, and was thus a very attractive candidate for our implementation of the concurrent patricia trie.\cite{MozillaResearch} Specifically, in terms of safety it guarantees no data races, buffer or stack overflow, and null pointer exceptions for most use cases. Due to its heavily static nature, it can validate the compiled program to be free of such errors and thus allows us to leverage this in creating a concurrent Patricia trie that has more simple aspects of memory management handled for us. Similarly, it is a language that focuses on speed, which will also help us achieve our desired result of having a Patricia trie implementation that out-performs its counterparts.
\par
However, Rust's approach to memory management, namely it being handled almost entirely by the language without a separate runtime for a garbage collector, has its limits. While this is true for most use cases, it becomes increasingly difficult for the compiler to guarantee this when the complexities of parallel lock-free algorithms come into play. The very nature of these algorithms relies more heavily on complex linearizability reasoning than the compiler can deduce, which means that more complex concurrency problems must have their own memory management, as is in our case.
\par
Since we are building a lock-free data structure with managed memory instead of using a garbage collector, we ended up rolling our own HazardPointer class that will allow us to safely manage memory with no runtime exceptions. The Rust standard library, at the time of this paper, did not have any sort of HazardPointer class for lock-free memory management, which was disappointing since it prides itself on being highly parallelizable and safe. Our implementation is heavily based off one we found implemented for a concurrent hash array mapped trie that was implemented in Rust\cite{CHAMT}, but the code needed to be modified to fit our use case and also because it was no longer properly compiling. This implementation is heavily based on the seminal hazard pointer paper by Michael \cite{Michael2004}, and utilizes recent research in efficient CAS contention management for further improvements.\cite{Dice2013}
\par


\section{Algorithm Description}
Our algorithm is devised from two state of the art techniques for concurrent trees both relying on CAS operations. The two techniques we will be using are the generalized non-blocking tree approach using flags inside nodes \cite{Shafiei2013,Brown2014} and the similar technique to have the flags be inside the edges instead of the nodes. \cite{Natarajan2014} By combining these two methods we will be able to take the advantages of both with some minor drawbacks.
\par
The main advantage to come the edge-based technique is that it allows for insert and delete operations to flag a smaller portion of the tree and use fewer atomic instructions. \cite{Natarajan2014} By doing this, the implementation will create fewer conflicts and will allow for more concurrent insert and delete operations, allowing for great speedups for trees with heavy emphasis on insert and delete. However, the method also causes seek operations to perform slightly worst sometimes by increasing the time it takes for them to reach their target node when there is a marked branch in their path. Because of this, we will combinen this approach with the more tradidional node-based approach for certain operations so that we can further improve the edge-based approach by only using it when necessary. 

\subsection{Data Structures}
The data structure used to store the Patricia Tree is simply a tree that stores a grand root node and a logical root node as well as all the functions to search and modify inside the tree. The two root structure is used for creating searching by forcing enough nodes to make a proper seek record.
\par
The individual node structure is also relatively simple. It contains a key to store for comparison purposes, which is technically uneeded due to the nature of a trie, but very helpful. It contains a value and a mask, and perhaps the most important part is that instead of classic pointers to the children, it makes use of an AtomicStampedReference that allows for CAS operations on the edge between the two nodes and allows for both a node pointer and an integer to be stored. By allowing the integer to be stored inside the edge, it enables the edge to be flagged on the edge and is a large part of how our algorithm works.


\subsection{Algorithm Basics}


\section{Conclusion}
The conclusion goes here, once we finish the above sections.


\section{Acknowledgment}
We would like to thank Niloufar Shafiei for providing her source code for her paper and her willigness to help us.


% references section
\bibliographystyle{IEEEtran}
\bibliography{IEEEabrv,references}


\end{document}
